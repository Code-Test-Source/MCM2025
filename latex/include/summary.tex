\begin{abstract}
    
    The Olympic Games, inspired by the ancient Greek Games, is the world's premier international multi-sport event. Held every four years, it features summer and winter editions, bringing together athletes from around the globe to compete in a wide range of sports.
    For every sports lover, the Stimulation of Olympic Medals is a very interesting topic, while it's crucial for countries to adjust their strategy. 
    \textbf{In this paper, we will analyze the data of the Summer Olympic Games from 1896 to 2024, and use the data to simulate the results of the 2028 and 2032 Olympics using different methods from `PRE' model mostly based on machine learning.}

    First, we've examined, cleaned and transformed the raw data, setting the ice sports aside, combining various teams of one country, and mapping the countries which no longer exist to the current existing country.
    Then, we turn to machine learning, select the independent variables to use by \textbf{Correlation Coefficient Matrix} and \textbf{Principal Component Analysis}, split the data into training and testing parts, and use \textbf{eXtreme Gradient Boosting} methods to train features and target matrix. \textbf{Kolmogorov-Smirnov Test} is used to ensure the effectiveness of the model, while history data are reprocessed using feature importance and data in recent years are given higher weights to improve the accuracy of the model. In the end, we use the model to predict the results of the 2028 and 2032 Olympics, and the results are shown in multiple forms.

    After that, we encode country labels, create feature dataset of each country,and filter out data for countries that won medals in 1896 to construct training and testing datasets of the feature of `First Win Country'.\textbf{Random Forest} is used to train the model, and the results of the top ten countries most likely to win their first medal in 2028 are shown in the form of a bar chart.

    By analyzing the data given, we find that some specific sports and events play a significant role in the medal tally of some specific countries, e.g: long distance race for Kenya. We calculate the \textbf{proportion} of medals from these events, further estimate their importance, and finally come to the result of the extent to which choosing these events impact countries performance in the medal list.

    Finally, we combined methods used above and created a comprehensive prediction model called \textbf{`PRE'} model, naming after the primary methods we use. The model is specifically-tuned to calculate \textbf{Legendary Index} so as to detect \textbf{Great Couch Effect }using data of US gymnastics team coached by Bela Karolyi and Marta Karolyi. Lang Ping is successfully detected as a legendary coach, and we've found many more great teams and athletes, even controversial results(2024 Male Fencing,Italy) and decline of the great ranks(2024 Tennis,China).

    \textbf{Greatness is not born, but made.}We hope that our model may help countries' Olympic committes to accommodate strategies and achieve better results in the future Olympic Games.

    \begin{keywords}
    Olympic Games; Data Analysis; Stimulation; Machine Learning
    \end{keywords}
    
    \end{abstract}
    