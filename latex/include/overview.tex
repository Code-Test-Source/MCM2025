\section{Model Overview}

In our basic model, we aim at three goals: keeping the temperature as even as possible, making it close to the initial temperature and decreasing the water consumption.

We start with the simple sub-model where hot water is added constantly.
At first we introduce convection heat transfer control equations in rectangular coordinate system. Then we define the mean temperature of bath water.

Afterwards, we introduce Newton cooling formula to determine heat transfer
capacity. After deriving the value of parameters, we get calculating results via formula deduction and simulating results via CFD.

Secondly, we present the complicated sub-model in which hot water is
added discontinuously. We define an iteration consisting of two process:
heating and standby. As for heating process, we derive control equations and boundary conditions. As for standby process, considering energy conservation law, we deduce the relationship of total heat dissipating capacity and time.

Then we determine the time and amount of added hot water. After deriving the value of parameters, we get calculating results via formula deduction and simulating results via CFD.

At last, we define two criteria to evaluate those two ways of adding hot water. Then we propose optimal strategy for the user in a bathtub.
The whole modeling process can be shown as follows.

\begin{figure}[h] 
\centering
\includegraphics[width=12cm]{fig1.jpg}
\caption{Modeling process} \label{fig1}
\end{figure}