\newcounter{lastpage}
\setcounter{lastpage}{\value{page}}
\thispagestyle{empty} 

\section*{Report on Use of AI}

\begin{enumerate}
\item OpenAI ChatGPT ( Dec 14, 2024 version, ChatGPT-4o ) 
\begin{description}
\item[Query1:] Information on the 2028 and 2032 Olympic Games 
\item[Output:] The upcoming Summer Olympic Games are scheduled as follows:

2028 Summer Olympics – Los Angeles, USA

Los Angeles will host the 2028 Summer Olympics from July 14 to July 30, 2028. This marks the third time the city will host the Games, following the 1932 and 1984 Olympics. The 2028 Games will feature 35 sports, including the debut of flag football and squash, and the return of cricket and lacrosse. Notably, the Games will utilize existing venues, with no new permanent structures being built, aiming for a sustainable and cost-effective event. 
WIKIPEDIA

2032 Summer Olympics – Brisbane, Australia

Brisbane, Queensland, Australia, is set to host the 2032 Summer Olympics from July 23 to August 8, 2032. This will be the third time Australia hosts the Summer Olympics, following Melbourne in 1956 and Sydney in 2000. The Games will feature 35 sports, with cricket making its return to the Olympics for the first time since 1900. The event will utilize existing venues, with plans for upgrades and new facilities to enhance the city's infrastructure. 
WIKIPEDIA

Both Games are expected to showcase a blend of traditional and new sports, reflecting the evolving nature of the Olympic program.
\end{description}
\item Github Copilot ( 2024 version, ChatGPT-4o, o1( preview ) ) 
\begin{description}
\item[Query1:] Python grammar and other machine learning libraries 
\item[Output:] Python is a versatile programming language widely used in machine learning (ML) and natural language processing (NLP) due to its simplicity and the extensive ecosystem of libraries it offers. Here are some notable Python libraries in these domains:

Machine Learning Libraries:

scikit-learn: A comprehensive library for data mining and data analysis, providing simple and efficient tools for predictive data analysis. It supports various supervised and unsupervised learning algorithms. 
WIKIPEDIA

TensorFlow: An open-source framework developed by Google for building and deploying machine learning models, particularly deep learning models. It offers a flexible ecosystem for research and production.

Keras: A high-level neural networks API, written in Python and capable of running on top of TensorFlow. It allows for easy and fast prototyping of deep learning models.

PyTorch: An open-source machine learning library developed by Facebook's AI Research lab. It provides a flexible platform for building and training deep learning models.

XGBoost: A scalable and efficient implementation of gradient boosting framework, widely used for structured/tabular data.
\end{description}

\item Deepseek ( 2025 version )
\begin{description}
    \item[Query1:] Find some references about Medal prediction in the Olympic Games 
    \item[Output:] 
    Here are some references related to medal prediction in the Olympic
Games, focusing on methodologies, data-driven approaches, and key
insights from academic research and practical applications:

\begin{center}\rule{0.5\linewidth}{0.5pt}\end{center}

\hypertarget{academic-papers-on-medal-prediction}{%
\subsubsection*{\texorpdfstring{\textbf{1. Academic Papers on Medal
Prediction}}{1. Academic Papers on Medal Prediction}}\label{academic-papers-on-medal-prediction}}

\hypertarget{predicting-olympic-medal-counts-the-effects-of-economic-development-on-olympic-success}{%
\paragraph{\texorpdfstring{\textbf{1.1. ``Predicting Olympic Medal
Counts: The Effects of Economic Development on Olympic
Success''}}{1.1. ``Predicting Olympic Medal Counts: The Effects of Economic Development on Olympic Success''}}\label{predicting-olympic-medal-counts-the-effects-of-economic-development-on-olympic-success}}

\begin{itemize}
\item
  \textbf{Authors}: Daniel K. N. Johnson, Ayfer Ali
\item
  \textbf{Journal}: \emph{Social Science Quarterly} (2004)
\item
  \textbf{Summary}: This paper explores the relationship between a
  country's economic development (GDP, population) and its Olympic medal
  count. It uses regression analysis to predict medal totals based on
  economic indicators.
\item
  \textbf{Key Insight}: Economic factors (GDP per capita, population
  size) are strong predictors of Olympic success.
\item
  \textbf{Link}:
  \href{https://doi.org/10.1111/j.0038-4941.2004.08501009.x}{DOI:10.1111/j.0038-4941.2004.08501009.x}
\end{itemize}

\begin{center}\rule{0.5\linewidth}{0.5pt}\end{center}

\hypertarget{a-machine-learning-approach-to-predicting-olympic-medal-counts}{%
\paragraph{\texorpdfstring{\textbf{1.2. ``A Machine Learning Approach to
Predicting Olympic Medal
Counts''}}{1.2. ``A Machine Learning Approach to Predicting Olympic Medal Counts''}}\label{a-machine-learning-approach-to-predicting-olympic-medal-counts}}

\begin{itemize}
\item
  \textbf{Authors}: S. S. R. Abidi, M. Manzoor
\item
  \textbf{Conference}: \emph{International Conference on Machine
  Learning and Applications} (2018)
\item
  \textbf{Summary}: This study applies machine learning algorithms
  (e.g., Random Forest, Gradient Boosting) to predict Olympic medal
  counts using historical data, economic indicators, and sports-specific
  features.
\item
  \textbf{Key Insight}: Machine learning models outperform traditional
  regression models in accuracy, especially when incorporating
  sport-specific data.
\item
  \textbf{Link}:
  \href{https://ieeexplore.ieee.org/document/8649995}{IEEE Xplore}
\end{itemize}

\begin{center}\rule{0.5\linewidth}{0.5pt}\end{center}

\hypertarget{the-role-of-host-nation-advantage-in-olympic-success}{%
\paragraph{\texorpdfstring{\textbf{1.3. ``The Role of Host Nation
Advantage in Olympic
Success''}}{1.3. ``The Role of Host Nation Advantage in Olympic Success''}}\label{the-role-of-host-nation-advantage-in-olympic-success}}

\begin{itemize}
\item
  \textbf{Authors}: Andrew B. Bernard, Meghan R. Busse
\item
  \textbf{Journal}: \emph{The Review of Economics and Statistics} (2004)
\item
  \textbf{Summary}: This paper investigates the ``host nation
  advantage'' and its impact on medal counts. It uses historical data to
  quantify the boost in performance for host countries.
\item
  \textbf{Key Insight}: Host nations typically win an additional 1.3\%
  of total medals, with the effect lasting for two Olympic cycles.
\item
  \textbf{Link}:
  \href{https://doi.org/10.1162/003465304323023723}{DOI:10.1162/003465304323023723}
\end{itemize}

\begin{center}\rule{0.5\linewidth}{0.5pt}\end{center}

\hypertarget{data-driven-approaches}{%
\subsubsection*{\texorpdfstring{\textbf{2. Data-Driven
Approaches}}{2. Data-Driven Approaches}}\label{data-driven-approaches}}

\hypertarget{gracenote-olympic-medal-predictions}{%
\paragraph{\texorpdfstring{\textbf{2.1. ``Gracenote Olympic Medal
Predictions''}}{2.1. ``Gracenote Olympic Medal Predictions''}}\label{gracenote-olympic-medal-predictions}}

\begin{itemize}
\item
  \textbf{Source}: Gracenote Sports
\item
  \textbf{Summary}: Gracenote uses a proprietary algorithm to predict
  Olympic medal counts based on historical performance, recent results,
  and athlete rankings. Their predictions are widely cited in the media.
\item
  \textbf{Key Insight}: The model incorporates both individual athlete
  performance and team dynamics.
\item
  \textbf{Link}:
  \href{https://www.gracenote.com/sports/olympics/}{Gracenote Sports}
\end{itemize}

\begin{center}\rule{0.5\linewidth}{0.5pt}\end{center}

\hypertarget{nielsens-olympic-performance-predictions}{%
\paragraph{\texorpdfstring{\textbf{2.2. ``Nielsen's Olympic Performance
Predictions''}}{2.2. ``Nielsen's Olympic Performance Predictions''}}\label{nielsens-olympic-performance-predictions}}

\begin{itemize}
\item
  \textbf{Source}: Nielsen Sports
\item
  \textbf{Summary}: Nielsen uses a combination of historical data,
  athlete rankings, and economic indicators to predict medal counts for
  each country.
\item
  \textbf{Key Insight}: The model highlights the importance of recent
  performance trends in predicting future success.
\item
  \textbf{Link}: \href{https://www.nielsen.com/sports/}{Nielsen Sports}
\end{itemize}

\begin{center}\rule{0.5\linewidth}{0.5pt}\end{center}

\hypertarget{practical-applications-and-case-studies}{%
\subsubsection*{\texorpdfstring{\textbf{3. Practical Applications and
Case
Studies}}{3. Practical Applications and Case Studies}}\label{practical-applications-and-case-studies}}

\hypertarget{predicting-olympic-medal-counts-using-bayesian-hierarchical-models}{%
\paragraph{\texorpdfstring{\textbf{3.1. ``Predicting Olympic Medal
Counts Using Bayesian Hierarchical
Models''}}{3.1. ``Predicting Olympic Medal Counts Using Bayesian Hierarchical Models''}}\label{predicting-olympic-medal-counts-using-bayesian-hierarchical-models}}

\begin{itemize}
\item
  \textbf{Authors}: Andrew Gelman, Jennifer Hill
\item
  \textbf{Book}: \emph{Data Analysis Using Regression and
  Multilevel/Hierarchical Models} (2006)
\item
  \textbf{Summary}: This book includes a case study on predicting
  Olympic medal counts using Bayesian hierarchical models, which account
  for country-level and sport-level variability.
\item
  \textbf{Key Insight}: Hierarchical models provide more accurate
  predictions by capturing the nested structure of Olympic data (e.g.,
  athletes within countries, sports within events).
\item
  \textbf{Link}:
  \href{https://www.cambridge.org/9780521686891}{Cambridge University
  Press}
\end{itemize}

\begin{center}\rule{0.5\linewidth}{0.5pt}\end{center}

\hypertarget{the-economics-of-the-olympics-predicting-medal-counts}{%
\paragraph{\texorpdfstring{\textbf{3.2. ``The Economics of the Olympics:
Predicting Medal
Counts''}}{3.2. ``The Economics of the Olympics: Predicting Medal Counts''}}\label{the-economics-of-the-olympics-predicting-medal-counts}}

\begin{itemize}
\item
  \textbf{Authors}: Stefan Szymanski
\item
  \textbf{Journal}: \emph{Journal of Sports Economics} (2010)
\item
  \textbf{Summary}: This paper examines the economic and demographic
  factors that influence Olympic success, using regression analysis to
  predict medal counts.
\item
  \textbf{Key Insight}: Population size and GDP are the most significant
  predictors, but sport-specific investments also play a role.
\item
  \textbf{Link}:
  \href{https://doi.org/10.1177/1527002510368101}{DOI:10.1177/1527002510368101}
\end{itemize}

\begin{center}\rule{0.5\linewidth}{0.5pt}\end{center}

\hypertarget{open-data-sources-for-medal-prediction}{%
\subsubsection*{\texorpdfstring{\textbf{4. Open Data Sources for Medal
Prediction}}{4. Open Data Sources for Medal Prediction}}\label{open-data-sources-for-medal-prediction}}

\hypertarget{olympic-medal-dataset}{%
\paragraph{\texorpdfstring{\textbf{4.1. Olympic Medal
Dataset}}{4.1. Olympic Medal Dataset}}\label{olympic-medal-dataset}}

\begin{itemize}
\item
  \textbf{Source}: Kaggle
\item
  \textbf{Summary}: A comprehensive dataset containing historical
  Olympic medal counts, athlete information, and event results.
\item
  \textbf{Link}:
  \href{https://www.kaggle.com/datasets/heesoo37/120-years-of-olympic-history-athletes-and-results}{Kaggle
  Olympic Dataset}
\end{itemize}

\begin{center}\rule{0.5\linewidth}{0.5pt}\end{center}

\hypertarget{world-bank-economic-indicators}{%
\paragraph{\texorpdfstring{\textbf{4.2. World Bank Economic
Indicators}}{4.2. World Bank Economic Indicators}}\label{world-bank-economic-indicators}}

\begin{itemize}
\item
  \textbf{Source}: World Bank
\item
  \textbf{Summary}: Provides GDP, population, and other economic
  indicators that can be used as predictors in medal count models.
\item
  \textbf{Link}: \href{https://data.worldbank.org/}{World Bank Data}
\end{itemize}

    \end{description}

\item Doubao ( 2025 version ) 
\begin{description}
\item[Query1:] The year Bela and Marta instructed the us gym team 
\item[Output:] Bela and Marta instructed the US gym team in the following years:

- \textbf{1981}: They defected to the US and began coaching gymnastics soon after.

- \textbf{1984}: They coached Mary Lou Retton, who won America's first gold medal in the Olympic all - arounds.

- \textbf{1991}: They trained Kim Zmeskal, who became the world champion.

- \textbf{1996}: Marta coached the "Magnificent Seven" member Kerri Strug to win the Olympic team gold.

- \textbf{2000}: Bela was named national team coordinator.

- \textbf{2001 - 2016}: Marta served as the coordinator of the US women's gymnastics national team.

\end{description}
\end{enumerate}
