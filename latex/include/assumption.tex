
\section{Assumptions and Justification}

To simplify the problem, make it convenient for us to construct simulation model and ensure the effectiveness of the model, we make the following basic assumptions, each of which is properly justified.

\begin{itemize}
\item {\bf Data preprocess}. We've examined that most of the data provided by the problem is accurate and reliable, although some may not be the same as other data. However, in order to provide an effective and accurate model, we will preprocess the data to make it more suitable.We set the ice sports aside, combine various teams of one country, and map the countries which no longer exist to the current existing country 

\item {\bf All the physical properties of bath water, bathtub and air are assumed to be stable}. The change of those properties like specific heat, thermal conductivity and density is rather small according to some studies \cite{5}. It is complicated and unnecessary to consider these little change so we ignore them.

\item {\bf There is no internal heat source in the system consisting of bathtub, hot water and air}. Before the person lies in the bathtub, no internal heat source exist except the system components. The circumstance where the person is in the bathtub will be investigated in our later discussion.

\item {\bf We ignore radiative thermal exchange}. According to Stefan-Boltzmann’s law, the radiative thermal exchange can be ignored when the temperature is low. Refer to industrial standard \cite{6}, the temperature in bathroom is lower than 100 $^{\circ}$C, so it is reasonable for us to make this assumption.

\item {\bf The temperature of the adding hot water from the faucet is stable}. This hypothesis can be easily achieved in reality and will simplify our process of solving the problem.
\end{itemize}
