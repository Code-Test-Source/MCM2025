\section{Introduction}

\subsection{Background and Literature Review}

\subsection{Restatement of the Problem}
Considering the background, in this paper we are required to solvye the following problems:

\textbf{Task 1:} Develop a model for medal counts for each country, both \textbf{Gold} and \textbf{Total}, and use the model to predict various countries' performance in the 2028 and 2032 Olympics. The model also include estimates of the precision and measures of how well the model performs.

\textbf{Task 2:} Develop a model for prediciton of when a country will win its \textbf{first} medal in the Olympics, and the probabililty of winning it in the coming next LA Olympics. Also, we will evaluate this model.

\textbf{Task 3:} Develop a model for estimation of the \textbf{"great coach"} effect, and further using this model to identify three countries suitable for imitating this strategy.

\textbf{Task 4:} Calculating the \textbf{"great athlete"} effect, referring to the phenomenon that some great athletes won a large number of the (Gold) medals of a certain sport or event, and sometimes his/her country's medal tally greatly depend on him/her during his/her athlete career. We will also explain how this can inform country Olympic committees.


In order to solve those problems, we will proceed as follows:

\begin{itemize}
\item {\bf Stating assumptions}. By stating our assumptions, we will narrow the focus of our approach towards the problems and provide some insight into bathtub water temperature issues.

\item {\bf Making notations}. We will give some notations which are important for us to clarify our models.

\item {\bf Presenting our model}. In order to investigate the problem deeper, we divide our model into two sub-models. One is a steady convection heat transfer sub-model in which hot water is added constantly. The other one is an unsteady convection heat transfer sub-model where hot water is added discontinuously.

\item {Defining evaluation criteria and comparing sub-models}. We define two main criteria to evaluate our model: the mean temperature of bath water and the amount of inflow water.

\item {\bf Analysis of influencing factors}. In term of the impact of different factors on our model, we take those into consideration: the shape and volume of the tub, the shape/volume/temperature of the person in the bathtub, the motions made by the person in the bathtub and adding a bubble bath additive initially.

\item {\bf Model testing and sensitivity analysis}. With the criteria defined before, we evaluate the reliability of our model and do the sensitivity analysis.

\item {\bf Further discussion}. We discuss about different ways to arrange inflow faucets. Then we improve our model to apply them in reality.

\item {\bf Evaluating the model}. We discuss about the strengths and weaknesses of our model:

\begin{itemize}
\item[1)] ... 
\item[2)] ...
\item[3)] ...
\item[4)] ...
\end{itemize}

\end{itemize}

\subsection{Our Work}