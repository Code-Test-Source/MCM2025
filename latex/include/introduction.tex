\section{Introduction}

\subsection{Background and Literature Review}

\subsection{Restatement of the Problem}

We are required to establish a model to determine the change of water temperature in space and time. Then we are expected to propose the best strategy for the person in the bathtub to keep the water temperature close to initial temperature and even throughout the tub. Reduction of waste of water is also needed. In addition, we have to consider the impact of different conditions on our model, such as different shapes and volumes of the bathtub, etc.

In order to solve those problems, we will proceed as follows:

\begin{itemize}
\item {\bf Stating assumptions}. By stating our assumptions, we will narrow the focus of our approach towards the problems and provide some insight into bathtub water temperature issues.

\item {\bf Making notations}. We will give some notations which are important for us to clarify our models.

\item {\bf Presenting our model}. In order to investigate the problem deeper, we divide our model into two sub-models. One is a steady convection heat transfer sub-model in which hot water is added constantly. The other one is an unsteady convection heat transfer sub-model where hot water is added discontinuously.

\item {Defining evaluation criteria and comparing sub-models}. We define two main criteria to evaluate our model: the mean temperature of bath water and the amount of inflow water.

\item {\bf Analysis of influencing factors}. In term of the impact of different factors on our model, we take those into consideration: the shape and volume of the tub, the shape/volume/temperature of the person in the bathtub, the motions made by the person in the bathtub and adding a bubble bath additive initially.

\item {\bf Model testing and sensitivity analysis}. With the criteria defined before, we evaluate the reliability of our model and do the sensitivity analysis.

\item {\bf Further discussion}. We discuss about different ways to arrange inflow faucets. Then we improve our model to apply them in reality.

\item {\bf Evaluating the model}. We discuss about the strengths and weaknesses of our model:

\begin{itemize}
\item[1)] ... 
\item[2)] ...
\item[3)] ...
\item[4)] ...
\end{itemize}

\end{itemize}

\subsection{Our Work}