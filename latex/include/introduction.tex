\section{Introduction}

\subsection{Background and Literature Review}

The Olympic Games, often simply referred to as the Olympics, are the world's foremost international multi-sport events. With a history that dates back over 2,000 years to ancient Greece, the modern Olympics were revived in 1896 by Pierre de Coubertin.
The Olympics are held every four years, with the Summer Olympics and the Winter Olympics alternating every two years. The Summer Olympics feature a vast array of sports, from athletics and swimming, which are considered the cornerstones of the Games, to more specialized sports like fencing, badminton, and gymnastics. Athletes from around the globe gather to compete at the highest level, showcasing their extraordinary skills, determination, and physical prowess.

Over the years, the Olympics have become a platform for athletes to break records, inspire generations, and promote cultural exchange. They have also had a significant impact on the host cities, driving urban development, improving infrastructure, and boosting the local economy. Despite facing various challenges, including political issues and the impact of global events, the Olympic Games continue to hold a special place in the hearts of people worldwide, symbolizing the power of sports to unite and uplift humanity.

The prediction of Olympic medal counts has long been a topic of interest among sports enthusiasts, statisticians, and researchers. Understanding how to forecast the number of medals a country or athlete might win is not only a matter of curiosity but also has implications for sports management, marketing, and national pride. 

Traditional approaches are typically made closer to the start of an upcoming Olympic Games when information about the current athletes scheduled to compete becomes available. This approach allows for a more accurate assessment of a country's or athlete's medal prospects. For example, the virtual medal table forecast by Nielsen \cite{1} provides a more real-time and data-driven prediction. By incorporating current athlete performance, injuries, and recent competition results, these modern models can better capture the dynamic nature of sports.

However, these data may be concealed and intentionally modified by countries' Olympic committees, which may produce misleading predictions.As a result, our model will be based on the historical data of the Olympic Games, which is more reliable and less likely to be manipulated.
Research in this area has also explored the use of advanced statistical techniques. Machine learning algorithms, for instance, have been employed to analyze large datasets encompassing a wide range of variables related to athletes, sports, and countries. These algorithms can identify complex patterns and relationships that might not be apparent through traditional statistical methods.

In conclusion, the field of predicting Olympic medal counts has evolved significantly over the years. While historical contemporary methods still provide some basis for understanding trends, the focus has shifted towards historical data. The use of advanced statistical techniques and an increased awareness of external factors have improved the accuracy of these predictions. 
However, there is still room for further research, particularly in the use of machine learning methods and elements that can't be captured directly by historical data, such as the Great Coach Effect. As the Olympics continue to evolve, so too will the methods and models used to predict the medal counts, ensuring that this remains a vibrant and relevant area of study.

\subsection{Restatement of the Problem}
Considering the background, in this paper we are required to solvye the following problems:

\textbf{Task 1:} Develop a model for medal counts for each country, both \textbf{Gold} and \textbf{Total}, and use the model to predict various countries' performance in the 2028 and 2032 Olympics. The model also include estimates of the precision and measures of how well the model performs.

\textbf{Task 2:} Develop a model for prediciton of when a country will win its \textbf{first} medal in the Olympics, and the probabililty of winning it in the coming 2028 Los Angeles Olympics. Also, we will evaluate this model.

\textbf{Task 3:} Develop a model for estimation of the \textbf{"great coach"} effect, and further using this model to identify three countries suitable for imitating this strategy.

\textbf{Task 4:} Calculating the \textbf{"great athlete"} effect, referring to the phenomenon that some great athletes won a large number of the (Gold) medals of a certain sport or event, and sometimes his/her country's medal tally greatly depend on him/her during his/her athlete career. We will also explain how this can inform country Olympic committees.


\subsection{Our Work}


In order to solve those problems, we will proceed as follows:

\begin{itemize}
\item {\bf Stating assumptions}. By stating our assumptions, we will narrow the focus of our approach towards the problems and provide some insight into bathtub water temperature issues.

\item {\bf Making notations}. We will give some notations which are important for us to clarify our models.

\item {\bf Presenting our model}. In order to investigate the problem deeper, we divide our model into two sub-models. One is a steady convection heat transfer sub-model in which hot water is added constantly. The other one is an unsteady convection heat transfer sub-model where hot water is added discontinuously.

\item {Defining evaluation criteria and comparing sub-models}. We define two main criteria to evaluate our model: the mean temperature of bath water and the amount of inflow water.

\item {\bf Analysis of influencing factors}. In term of the impact of different factors on our model, we take those into consideration: the shape and volume of the tub, the shape/volume/temperature of the person in the bathtub, the motions made by the person in the bathtub and adding a bubble bath additive initially.

\item {\bf Model testing and sensitivity analysis}. With the criteria defined before, we evaluate the reliability of our model and do the sensitivity analysis.

\item {\bf Further discussion}. We discuss about different ways to arrange inflow faucets. Then we improve our model to apply them in reality.

\item {\bf Evaluating the model}. We discuss about the strengths and weaknesses of our model:

\begin{itemize}
\item[1)] ... 
\item[2)] ...
\item[3)] ...
\item[4)] ...
\end{itemize}

\end{itemize}