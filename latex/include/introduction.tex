\section{Introduction}

\subsection{Background}

Bathing in a tub is a perfect choice for those who have been worn out after a long day's working. A traditional bathtub is a simply water containment vessel without a secondary heating system or circulating jets. Thus the temperature of water in bathtub declines noticeably as time goes by, which will influent the experience of bathing. As a result, the bathing person needs to add a constant trickle of hot water from a faucet to reheat the bathing water. This way of bathing will result in waste of water because when the capacity of the bathtub is reached, excess water overflows the tub.

An optimal bathing strategy is required for the person in a bathtub to get comfortable bathing experience while reducing the waste of water.

\subsection{Literature Review}

Korean physicist Gi-Beum Kim analyzed heat loss through free surface of water contained in bathtub due to conduction and evaporation \cite{1}. He derived a relational equation based on the basic theory of heat transfer to evaluate the performance of bath tubes. The major heat loss was found to be due to evaporation. Moreover, he found out that the speed of heat loss depends more on the humidity of the bathroom than the temperature of water contained in the bathtub. So, it is best to maintain the temperature of bathtub water to be between 41 to 45$^{\circ}$C and the humidity of bathroom to be 95\%.

When it comes to convective heat transfer in bathtub, many studies can be referred to. Newton's law of cooling states that the rate of heat loss of a body is proportional to the difference in temperatures between the body and its surroundings while under the effects of a breeze \cite{2}. Claude-Louis Navier and George Gabriel Stokes described the motion of viscous fluid substances with the Navier-Stokes equations. Those equations may be used to model the weather, ocean currents, water flow in a pipe and air flow around a wing \cite{3}.

In addition, some numerical simulation software are applied in solving and analyzing problems that involve fluid flows. For example, Computational Fluid Dynamics (CFD) is a common one used to perform the calculations required to simulate the interaction of liquids and gases with surfaces defined by boundary conditions \cite{4}.

\subsection{Restatement of the Problem}

We are required to establish a model to determine the change of water temperature in space and time. Then we are expected to propose the best strategy for the person in the bathtub to keep the water temperature close to initial temperature and even throughout the tub. Reduction of waste of water is also needed. In addition, we have to consider the impact of different conditions on our model, such as different shapes and volumes of the bathtub, etc.

In order to solve those problems, we will proceed as follows:

\begin{itemize}
\item {\bf Stating assumptions}. By stating our assumptions, we will narrow the focus of our approach towards the problems and provide some insight into bathtub water temperature issues.

\item {\bf Making notations}. We will give some notations which are important for us to clarify our models.

\item {\bf Presenting our model}. In order to investigate the problem deeper, we divide our model into two sub-models. One is a steady convection heat transfer sub-model in which hot water is added constantly. The other one is an unsteady convection heat transfer sub-model where hot water is added discontinuously.

\item {Defining evaluation criteria and comparing sub-models}. We define two main criteria to evaluate our model: the mean temperature of bath water and the amount of inflow water.

\item {\bf Analysis of influencing factors}. In term of the impact of different factors on our model, we take those into consideration: the shape and volume of the tub, the shape/volume/temperature of the person in the bathtub, the motions made by the person in the bathtub and adding a bubble bath additive initially.

\item {\bf Model testing and sensitivity analysis}. With the criteria defined before, we evaluate the reliability of our model and do the sensitivity analysis.

\item {\bf Further discussion}. We discuss about different ways to arrange inflow faucets. Then we improve our model to apply them in reality.

\item {\bf Evaluating the model}. We discuss about the strengths and weaknesses of our model:

\begin{itemize}
\item[1)] ... 
\item[2)] ...
\item[3)] ...
\item[4)] ...
\end{itemize}

\end{itemize}